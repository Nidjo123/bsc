\documentclass[times, utf8, zavrsni]{fer}
\usepackage{booktabs}

\begin{document}

% TODO: Navedite broj rada.
\thesisnumber{000}

\title{Gusta stereoskopska rekonstrukcija poluglobalnim podudaranjem}

\author{Nikola Bunjevac}

\maketitle

% Ispis stranice s napomenom o umetanju izvornika rada. Uklonite naredbu \izvornik ako želite izbaciti tu stranicu.
\izvornik

% Dodavanje zahvale ili prazne stranice. Ako ne želite dodati zahvalu, naredbu ostavite radi prazne stranice.
\zahvala{}

\tableofcontents

\chapter{Uvod}
Računalni vid iznimno je zanimljivo interdisciplinarno područje koje se svrstava kao grana umjetne inteligencije.
Ono se bavi omogućavanjem računalima da shvate i interpretiraju podatke iz digitalnih slika i videozapisa.
Vid je jedno od najznačajnijih osjetila jer pomoću njega primamo mnoštvo korisnih informacija.
Pomoću vida se orijentiramo u prostoru, prepoznajemo druge ljude i njihova lica, čitamo itd.
Kada bi računala mogla interpretirati vizualne podražaje poput ljudi, to bi omogućilo velik
napredak u umjetnoj inteligenciji. Nažalost, to je još uvijek neriješen problem. Cijelo područje
je nastalo šezdesetih godina prošlog stoljeća kada se mislilo da će se taj problem relativno brzo riješiti.
Naime, profesor je kao zadatak studentu zadao da na robota stavi kameru i da robot opisuje ono što vidi.
Vrlo brzo se pokazalo kako je problem puno složeniji nego što se mislilo.
S druge strane, od tada su ostvareni veliki napretci u tom području kao i općenito u umjetnoj inteligenciji.

$$ f(x) = x^2 + y_1 $$

\chapter{Stereoskopski sustavi}

\chapter{Lokalne metode}

\chapter{Poluglobalno podudaranje}

\chapter{Rezultati}

\chapter{Zaključak}
Zaključak.

\bibliography{literatura}
\bibliographystyle{fer}

\begin{sazetak}
  Sažetak na hrvatskom jeziku.

\kljucnerijeci{Ključne riječi, odvojene zarezima.}
\end{sazetak}

% TODO: Navedite naslov na engleskom jeziku.
\engtitle{Dense stereoscopic reconstruction with semi-global matching}
\begin{abstract}
Abstract.

\keywords{Keywords.}
\end{abstract}

\end{document}
